% small.tex
\documentclass[12pt]{beamer}
\usetheme{amcg}
\beamertemplatenavigationsymbolsempty
\renewcommand{\thefootnote}{}
\providecommand{\e}[1]{\ensuremath{\times 10^{#1}}}
\usepackage{mathptmx}
\usepackage{helvet}
\newcommand\TILDE{\char`\~}
\usepackage{listings}

% items enclosed in square brackets are optional; explanation below
\title[Parallel]{Running Fluidity in Parallel}
\subtitle[]{}
\institute{1 - Dept of Earth Science and Engineering, Imperial College London}
\author[Jon Hill]{\large{Jon Hill}\inst{1}}
\date{}


\begin{document}


%--- the titlepage frame -------------------------%
\begin{frame}
  \titlepage
\end{frame}

%-- Overview slide --- %
\section*{Outline}
\begin{frame}
  \frametitle{Outline}
  \tableofcontents
\end{frame}

\section{Pre-running}
\begin{frame}
    \frametitle{FLML}
No changes required!
\vspace{5mm}

[Optional]
\begin{itemize}
\item Remove fields from stat file
\item Remove some fields from VTU
\end{itemize}
\end{frame}

\begin{frame}
    \frametitle{Decompose mesh}
\begin{itemize}
\item fldecomp
\item flderecomp
\end{itemize}
\vspace{5mm}

Both make with \texttt{make fltools}. Programs will be in bin
\end{frame}

\begin{frame}
    \frametitle{\texttt{fldecomp}}

\texttt{fldecomp -n 8 MeshName}
\vspace{5mm}

Will decompose mesh into 8 peices.
\end{frame}

\begin{frame}
    \frametitle{\texttt{flredecomp}}
\texttt{flredecomp -i 2 -n 8 InputMeshName OutputMeshName}
\vspace{5mm}

Will decompose mesh from 2 to 8.
\vspace{5mm}

\texttt{flredecomp -i 8 -n 2 InputMeshName OutputMeshName}
\vspace{5mm}

Will decompose mesh from 8 to 2.
\vspace{5mm}
Both need running on 8 processors
\end{frame}

\section{Running}
\begin{frame}
    \frametitle{Local systems}

\texttt{mpiexec -n 8 ../../bin/fluidity my.flml}
\end{frame}

\begin{frame}
    \frametitle{HECToR}

\texttt{module swap PrgEnv-cray PrgEnv-fluidity}
\vspace{5mm}

Submit job to back-end.
\vspace{5mm}

\texttt{qsub myscript.sct}
\end{frame}

\begin{frame}[fragile]
    \frametitle{HECToR}
\vspace{-2mm}
\lstset{language=bash,basicstyle=\tiny}
\begin{lstlisting}[language=bash]
#!/bin/bash --login
#PBS -N fluidity_run
#PBS -l mppwidth=512
#PBS -l mppnppn=4
#PBS -l walltime=0:10:00
#PBS -A n04-IC

module swap PrgEnv-cray PrgEnv-fluidity

# Change to the directory that the job was submitted from
cd $PBS_O_WORKDIR
 
# The following take a copy of the Fluidity Python directory and
# put it in the current directory. If we don't do this, we get import errors. 
export WORKING_DIR=$(pwd -P)
cp -r /usr/local/packages/fluidity/xe6/2.0/python/ .
export PYTHONPATH=$WORKING_DIR/python:$PYTHONPATH

# Set the number of MPI tasks
export NPROC=`qstat -f $PBS_JOBID | awk '/mppwidth/ {print $3}'`
# Set the number of MPI tasks per node
export NTASK=`qstat -f $PBS_JOBID | awk '/mppnppn/  {print $3}'`
aprun -n $NPROC -N $NTASK fluidity -l -v1  standing_wave.flml

# clean up the python directory
rm -rf python
\end{lstlisting}
\end{frame}

\begin{frame}
    \frametitle{HECToR: A run-through}
\begin{itemize}
\item Set-up on local machine
\item Copy to HECToR
\item Create PBS script
\item qsub
\item Copy results back to local machine \textbf{or} use paraview on HECToR
\end{itemize}
\end{frame}


\section{Post-running}
\begin{frame}
    \frametitle{Visualisation}
No different from serial - except .pvtu files, not .vtu
\vspace{5mm}

Log files (if used) will be one per processor.
\end{frame}


\end{document}

