%% Generated file to plot data from a datafile using pgfplots

%================================================================================
%|                                                                              |
%| If you plot a huge amount of data, and TeX runs out on its main memory,      |
%| you need to change its allowed maximum main memory in the texmf.cnf.         |
%| Google is your friend                                                        |
%|                                                                              |
%| The 'Preview package is used to remove white space around the tikzpicture    |
%| environment. If this fails for you, and/or alternatively you can comment     |
%| the relevant lines just above the '\begin{document}' command                |
%| and make use of the external tool pdfcrop.                                   |
%|                                                                              |
%================================================================================

\documentclass[11pt]{article}

% Set page legths
\special{papersize=50cm,50cm}
\hoffset-0.8in
\voffset-0.8in
\setlength{\paperwidth}{50cm}
\setlength{\paperheight}{50cm}
\setlength{\textwidth}{45cm}
\setlength{\textheight}{45cm}
\topskip0cm
\setlength{\headheight}{0cm}
\setlength{\headsep}{0cm}
\setlength{\topmargin}{0cm}
\setlength{\oddsidemargin}{0cm}
% set the pagestyle to empty (removing pagenumber etc)
\pagestyle{empty}

% load packages:
\usepackage{amsmath}
\usepackage{amssymb}
\usepackage{amstext}
\usepackage{amsfonts}

\usepackage{tikz}
\usepackage{pgfplots}
%\usepgfplotslibrary{external}
%\tikzexternalize

% Use newest spacing options (from v. 1.3 on)
\pgfplotsset{compat=newest}
\pgfplotsset{width=10cm}
\pgfplotsset{height=6cm}

\pgfplotsset{grid style={solid}}

% Remove white space from generated pdf,
% thus otaining a pdf with only the picture that can
% easily be included in a(nother) tex-document via the usual \includegraphic command.
% Benefit: 1. you can keep your pictures organised in a subfolder and
%          2. the picture remains a vector graphic :)
\usepackage[active, tightpage]{preview}
\PreviewEnvironment{tikzpicture}
\setlength\PreviewBorder{0pt}
% Alternatively, delete the three lines above and run 'pdfcrop filename.pdf',
% the result should be the same.

\begin{document}

\begin{center}

\begin{tikzpicture}
  \begin{semilogxaxis}[
        axis background/.style={fill=gray!10},
        axis x line*=bottom, axis y line*=left,
        % If more precision on x or y axis is needed, uncomment the relevant lines below:
        % scaled x ticks = false,
        % x tick label style={/pgf/number format/fixed, /pgf/number format/precision=3}, % 3 for 3 floating point digits
        % scaled y ticks = false,
        % y tick label style={/pgf/number format/fixed, /pgf/number format/precision=3}, % 3 for 3 floating point digits
        scale only axis, % might get 'dimension too large' error if switched on
        minor tick num=0,
        % restrict x to domain=-10:10, % use this if you get a 'dimension too large' error
        % restrict y to domain=-10:10, % use this if you get a 'dimension too large' error
        xlabel={Number of nodes},
        ylabel={$\alpha$},
        legend cell align=left, % best if aligned left
        legend style={legend pos=north east, %outer north east,
                      % specify legend entries:
                      % example: legend entries={entry1, entry2, entry3},
                      % if legend entries are too long, specify max text width/depth:
                      % nodes={text width=30pt,text depth=40},
                      % if you want to put the legend outside the figure envirmonent, do:
                      % legend to name=legendlabel,
                      % and then '\ref{legendlabel}' where you want the legend to appear
                      % don't forget to run pdflatex twice for it to pick up the changed reference!
                      font=\tiny},
        grid=major
        ]
    \addplot[color=red, solid, thick] plot coordinates { (5000,44.3468288202) (5000000,44.3468288202)};
    \addlegendentry{analytic};
    \addplot[color=blue, solid, line join=round, error bars/.cd, y dir=both, y explicit] table[x=num_nodes, y=angle_mean, y error=angle_std] {table-fixed/st-andrews-cross-validation_data.pgfdat};
    \addlegendentry{fixed mesh};
    \addplot[color=green!60!black, solid, line join=round, error bars/.cd, y dir=both, y explicit] table[x=num_nodes, y=angle_mean, y error=angle_std] {table-adaptive/st-andrews-cross-validation_data.pgfdat};
    \addlegendentry{adaptive mesh};
  \end{semilogxaxis}
\end{tikzpicture}

\end{center}
\end{document}
