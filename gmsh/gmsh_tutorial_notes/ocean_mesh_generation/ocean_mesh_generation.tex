\section{Ocean mesh generation}
\label{sect:ocean_mesh_generation}
In this section we briefly present the procedure of mesh generation on domains representative of realistic oceans, including shorelines. Clearly, the shorelines need to be extracted from a relevant source and be reconstructed in the Geometry module of Gmsh. For that purpose, Gmsh features the \emph{GSHHS plugin} \citep{lambrechts_et_al:2008}. As its name suggests, this plugin uses the \emph{Global Self-consistent, Hierarchical, High-resolution Shoreline Database}\footnote{available from \url{http://www.ngdc.noaa.gov/mgg/shorelines/gshhs.html}} \citep{wessel_smith:1996} as a source of shoreline contours. In specific, the plugin allows the user to select an area on the globe, and then generates a Gmsh geometry script fitting a spline to the shoreline points. The user can then draw open boundaries to form a closed domain. 
\par
The interested reader is deferred to \url{http://perso.uclouvain.be/jonathan.lambrechts/gmsh_ocean/} where screen-casts show how to use the plugin and generate meshes featuring higher resolution to towards the shoreline. However, the user can specify plugin options and then invoke plugins from within Gmsh scripts. Thus, a procedure where the user creates a series of scripts and semi-automates the procedure will be presented here in the near future.

%\par
%In this section we first show how to create geometries on a spherical manifold representative of shorelines. Then, combined with the methodology for drawing lines of constant longitude or constant latitude developed in the previous section, the reader will learn how to produce meshes on geometries representing realistic ocean domains.
%\par
%what is gshhs? how does one obtain it? Must give some configuration steps here
%\par
%What is the Gmsh GSHHS plugin? (no need for separate download, it comes with Gmsh!). Must give an overview of the plugin: its options and what they do.
%
%\subsection{Setting up the geometry}
%\label{ssect:setting_up}
%
%\subsubsection{Selecting the area on Earth to be meshed.}
%\par
%Show maps (reference Generic Mapping Tools), identify points, identify boxes, list point coordinates.
%
%\subsubsection{Some definitions}
%
%\subsubsection{Extracting the shoreline}
%
%\subsubsection{Drawing the open boundaries}
%
%\subsubsection{Specifying the mesh sizes}
%
%\subsection{Producing a mesh} 
