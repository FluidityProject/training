\section*{Summary}\small
This document is a tutorial on the Gmsh mesh generator. It is aimed towards complete beginners; only
some basic knowledge of the Linux terminal and a text editor is assumed. We first define what a mesh
is and then introduce the reader to the basics of the Gmsh graphical user interface. A basic,
two-dimensional, geometry is then constructed within Gmsh and a mesh is constructed. A more
complicated three-dimensional annulus is also constructed and meshed, demonstrating some
more advanced features of Gmsh.

Having mastered the basic usage of the graphical user interface, users are introduced to generating
simple meshes on the sphere. Finally, other tutorials and methds that show how to produce
meshes in realistic domains are briefly introduced in the last section.
%Knowledge is further built to produce meshes of realistic domains
%of the oceans to include boundaries extracted from a high resolution shorelines database.
