%-- Add sections and your outline will be created automatically --%
\subsection{Rotating periodic channel}

% Frame starts a new slide
\begin{frame}
    \frametitle{Rotating periodic channel}
\begin{itemize}
\item Unit square domain, periodic in zonal direction and zero-slip at North and South boundaries.  Coriolis forcing.
\item The flow is driven by a velocity source term:
\begin{equation*}
  \vec{F}=
  \begin{bmatrix}
    y^3 \\
    0
  \end{bmatrix}
\end{equation*}
\item Provides a convergence test for the $P_{1DG}P_2$ element pair.
\item A good example of using python state for online diagnostics and analysis, and also using python for setting initial conditions.
\item Run time: 10 min. 
\end{itemize}
\end{frame}
%
\begin{frame}
    \frametitle{Rotating periodic channel}
\begin{figure}
\includegraphics[width=0.6\textwidth]{./rotating_channel/analytic_solution}
\caption{Velocity forcing term and analytic solutions for velocity and pressure for the rotating periodic channel test case. Note that each of these quantities is constant in the x direction.}
\end{figure}
\end{frame}
%
\begin{frame}
    \frametitle{Rotating periodic channel}
\begin{figure}
\includegraphics[width=0.6\textwidth]{./rotating_channel/convergence}
\caption{Error in the pressure and velocity solutions for the rotating channel as a function of resolution.}
\end{figure}
\end{frame}
%
\begin{frame}
    \frametitle{Rotating periodic channel, exercises}
\begin{itemize}
\item Understand the use of analytic forcing functions in Fluidity using Python.
\item For the Continuous Galerkin example, see what the effect is of removing the SUPG stabilisation
\item Change the resolution of the adapted meshes
\item Change the spatial and temporal discretisations to get a less diffusive advection scheme
\end{itemize}
\end{frame}

