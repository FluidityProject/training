\documentclass[t]{beamer}
\usetheme{amcg}
\usepackage{hyperref}
\hypersetup{
    colorlinks=true,
    linkcolor=cyan,
    urlcolor=cyan
}
\usepackage{mathptmx}
\usepackage{helvet}
\newcommand\TILDE{\char`\~}
\usepackage{listings}
\setbeamertemplate{navigation symbols}{}

\author[Alexandros Avdis]{Applied Modelling and Computation Group\\[15pt]Alexandros Avdis}

\institute{Department of Earth Science and Engineering, Imperial College London}

\date{11-13 November 2015}
\title[What to do if your simulation breaks?]{What to do if your simulation breaks?}
\subtitle[]{Fluidity training workshop}

\setbeamersize{text margin left=0.3in}
\setbeamersize{text margin right=0.3in}

\begin{document}

\setbeamertemplate{footline}
{%
\begin{beamercolorbox}[ht=.35cm,dp=0.2cm,wd=\textwidth,leftskip=.3cm]{author in head/foot}%
        \begin{minipage}[c]{5cm}%
        \usebeamerfont{author in head/foot}
        \end{minipage}\hfill% 
        \begin{minipage}{6cm}
        \hfill\includegraphics[height=.5cm]{AMCGFlow-long}
        \end{minipage}
\end{beamercolorbox}%
}
\begin{frame}
\titlepage
\end{frame}

%%%%%%%%%%%%%%%%%%%%%%%%%%%%%%%%%%%%%%%%%%%%%%%%%%%%%%%%%

\begin{frame}{An overview to debugging your failing run.}

You \emph{will} encounter failures while running code.\\
\vspace{1ex}
You \emph{have already encountered} failing software: Crashes!\\
\vspace{1ex}
Keep calm and carry out the following steps:
\vspace{1ex}

\begin{itemize}
 \item[$\circ$] Collect information.
 \item[$\circ$] Interpret information: Think of a cause of the error.
 \item[$\circ$] Treat \& test.
 \item[$\circ$] Repeat, and remember simulation is an iterative process.
\end{itemize}

\end{frame}

%%%%%%%%%%%%%%%%%%%%%%%%%%%%%%%%%%%%%%%%%%%%%%%%%%%%%%%%%

\begin{frame}{Collect information.}

Collect and go through \emph{even if you do not understand all of it}\\
\vspace{20pt}
The model \& OS output is a primary source of information:
\begin{itemize}
 \item[$\circ$] log file (Fluidity options ``{\tt -l -v3}'')
 \item[$\circ$] error file (including backtrace)
 \item[$\circ$] pbs output (if running on a HPC)
 \item[$\circ$] stat file
 \item[$\circ$] vtus
\end{itemize}
\vspace{20pt}
Treat is as a puzzle/crossword: Go through the information again\\
\vspace{20pt}
Repeat \& Remember: Simulation is an iterative process.

\end{frame}

%%%%%%%%%%%%%%%%%%%%%%%%%%%%%%%%%%%%%%%%%%%%%%%%%%%%%%%%%

\begin{frame}{Further sources of useful information}

You can re-run a simulation in order to obtain more information:
\vspace{1ex}

\begin{itemize}
 \item[$\circ$] Re-run simulation with extra diagnostic fields.
 \item[$\circ$] Increase dump-rate.
 \item[$\circ$] Re-run simulation using binary compiled with debugging flags.
 \item[$\circ$] pbs output (if running on a HPC).
 \item[$\circ$] ``matrixdump'' and ``core'' files.
\end{itemize}

``matrixdump'' and ``core'' files are here included for completeness. However, they
large files, do not send to individuals/mailing lists. They are very useful when combined
with debuggers.

\end{frame}

%%%%%%%%%%%%%%%%%%%%%%%%%%%%%%%%%%%%%%%%%%%%%%%%%%%%%%%%%

\begin{frame}{Interpret information}

Going through the information will give you the time to think of what may have gone wrong.\\
\vspace{10pt}
Common causes of failing simulations:
\vspace{10pt}
\begin{itemize}
 \item[$\circ$] Bad environment variables, missing/out-of-date dependent software or modules
 \item[$\circ$] Bad options set-up:
 \begin{itemize}
  \item[$\circ$] Option errors
  \item[$\circ$] Faulty Python scripts
  \item[$\circ$] Under/over/badly constrained initial and/or boundary conditions
  \item[$\circ$] Numerical instability
  \item[$\circ$] Partitioning, adaptivity, field projection
 \end{itemize}
 \item[$\circ$] An error in software, a ``bug''.
 \begin{itemize}
  \item[$\circ$] When \emph{certain} of a bug, contact the development team.
 \end{itemize}
\end{itemize}

\end{frame}

%%%%%%%%%%%%%%%%%%%%%%%%%%%%%%%%%%%%%%%%%%%%%%%%%%%%%%%%%

\begin{frame}{Bad environment variables}

What to check:

\begin{itemize}
 \item[$\circ$] Check environment variables, in particular PYTHONPATH: type ``{\tt echo \$PYTHONPATH}'' into your terminal window
 \item[$\circ$] Check for absent dynamic libraries:
 \begin{itemize}
  \item[$\circ$] Navigate to the fluidity \texttt{bin} directory 
  \item[$\circ$] Issue \texttt{\$ ldd fluidity} at the prompt, it will list all dynamic libraries used by the binary.
  \item[$\circ$] There should be no missing libraries.
 \end{itemize}
\end{itemize}
\end{frame}

%%%%%%%%%%%%%%%%%%%%%%%%%%%%%%%%%%%%%%%%%%%%%%%%%%%%%%%%%

%\vspace{1ex}
\begin{frame}{Bad options set-up}
%If you're suspicious that your Fluidity build isn't working correctly, navigate to the Fluidity directory and type ``{\tt make unittest}''.
Option errors:
\vspace{1ex}
\begin{itemize}
 \item[$\circ$] The error message might contain advice on what to correct. Try again.
 \item[$\circ$] Use the {\tt diff} and {\tt meld} commands to compare against working {\tt .flml} files.
 \item[$\circ$] Regress to another, working set-up
\end{itemize}
\end{frame}

%%%%%%%%%%%%%%%%%%%%%%%%%%%%%%%%%%%%%%%%%%%%%%%%%%%%%%%%%

\begin{frame}{Regress}

Back-up your {\tt .flml} file and progressively remove fields:
\begin{itemize}
 \item[$\cdot$] Remove adaptivity.
 \item[$\cdot$] Remove any turbulence models and prescribe viscosity (as above)
 \item[$\cdot$] Remove parameterisations
 \item[$\cdot$] Reduce the number of spatial dimensions
 \item[$\cdot$] Simplify the geometry 
 \item[$\cdot$] Smooth BCs with discontinuities in space or time
 \item[$\cdot$] If you have a viscosity field, then set that to a value that gives you a Reynolds Number $\simeq 1000$
 \item[$\cdot$] Progressively remove prognostic fields
\end{itemize}
Then, add complexity back in \emph{\color{red}ONE STEP AT A TIME}

\end{frame}

%%%%%%%%%%%%%%%%%%%%%%%%%%%%%%%%%%%%%%%%%%%%%%%%%%%%%%%%%


\begin{frame}{Numerical instability}

The cause of numerical instability can be tricky to pin-point:
The symptom described within an error message may not directly relate the underlying cause of failure.

\vspace{1ex}
Check the various factors that can cause numerical instability:
\begin{itemize}
 \item[$\cdot$] Timestep, Courant Number, and temporal discretisation
 \item[$\cdot$] Spatial discretisation and element pair
 \item[$\cdot$] Mesh quality, Condition Number
 \item[$\cdot$] Mesh resolution, interpolation error, Grid Reynolds Number
 \item[$\cdot$] Field stabilisation, upwinding/slope-limiting
 \item[$\cdot$] Strong/weak boundary conditions
 \item[$\cdot$] Preconditioner, iterative solver, convergence criteria
\end{itemize}

\end{frame}

%%%%%%%%%%%%%%%%%%%%%%%%%%%%%%%%%%%%%%%%%%%%%%%%%%%%%%%%%

\begin{frame}{Monitoring the progress of a simulation}

If you are suspicious that your simulation may crash and you want to monitor its progress,
the following commands can be useful:

\vspace{1ex}
\begin{description}
 \item[] {\tt statplot *.stat} (press ``{\tt r}'' to refresh)
 \item[] {\tt tail -f fluidity.log-0}
 \item[] {\tt grep ``n/o iterations'' fluidity.log-0}
 \item[] {\tt grep reason fluidity.log-0}
\end{description}

For the last three to work, run with a verbosity of two or more, 
i.e. {\tt ./fluidity -v2 -l my.flml}

\end{frame}

%%%%%%%%%%%%%%%%%%%%%%%%%%%%%%%%%%%%%%%%%%%%%%%%%%%%%%%%%

\begin{frame}{A bug}

If you believe there might be a bug in the code
then your first step should be to contact the development team and submit a bug report.

\vspace{1ex}
The following third-party software can then be useful for slowly stepping through the simulation solve to determine the root cause:
\vspace{1ex}

\begin{itemize}
 \item[$\cdot$] Valgrind (checking for memory issues)
 \item[$\cdot$] GDB (a free Open Source de-bugger)
 \item[$\cdot$] DDT (a commercial alternative) 
\end{itemize}

Similarly ``vtune'' can be useful in determining which lines of code are taking up the most time in the solve.

\end{frame}

%%%%%%%%%%%%%%%%%%%%%%%%%%%%%%%%%%%%%%%%%%%%%%%%%%%%%%%%%

\begin{frame}{HELP!!!}

Consult the \href{https://fluidityproject.github.io/DocumentationSupport/index.html}{documentation}:
\begin{itemize}
 \item\href{https://github.com/FluidityProject/fluidity/wiki}{Fluidity wiki}
 \item The troubleshooting section of the \href{https://fluidityproject.github.io/DocumentationSupport/Manuals/index.html}{Fluidity manual}.
\end{itemize}

\vspace{1ex}
Fluidity has a very eager and responsive development community.  
We can be contacted either through the \href{mailman.ic.ac.uk/mailman/listinfo/fluidity}{mailing list}
or through IRC ({\tt \# amcg}).

\vspace{1ex}
It would be helpful for you to include information about:
\begin{description}
 \item[$\cdot$] What simulation are you trying to run?
 \item[$\cdot$] What equations are you solving?
 \item[$\cdot$] What discretisation are you using?
 \item[$\cdot$] What similar simulations do you have that {\it did} work?
 
\end{description}
\end{frame}
%%%%%%%%%%%%%%%%%%%%%%%%%%%%%%%%%%%%%%%%%%%%%%%%%%%%%%%%%

\begin{frame}{Questions?}
\begin{center}
\vfill
AMCG:\\
\url{http://amcg.ese.ic.ac.uk/}\\[15pt]
Fluidity:\\
\url{http://fluidityproject.github.io/}\\[15pt]
Fluidity code on GitHub:\\
\url{https://github.com/FluidityProject/fluidity}
\url{https://github.com/FluidityProject}\\[15pt]
Fluidity wiki:\\
\url{https://github.com/FluidityProject/fluidity/wiki}
\end{center}
\end{frame}

\end{document}
