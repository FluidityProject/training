\documentclass[t]{beamer}
\usetheme{amcg}
\usepackage{hyperref}
\hypersetup{
    colorlinks=true,
    linkcolor=cyan,
    urlcolor=cyan
}
\usepackage{mathptmx}
\usepackage{helvet}
\newcommand\TILDE{\char`\~}
\usepackage{listings}
\setbeamertemplate{navigation symbols}{}

\author[Alexandros Avdis]{Applied Modelling and Computation Group\\[15pt]Alexandros Avdis}

\institute{Department of Earth Science and Engineering, Imperial College London}

\date{11-13 November 2015}
\title[What to do if your simulation breaks?]{What to do if your simulation breaks?}
\subtitle[]{Fluidity training workshop}

\setbeamersize{text margin left=0.3in}
\setbeamersize{text margin right=0.3in}

\begin{document}

\setbeamertemplate{footline}
{%
\begin{beamercolorbox}[ht=.35cm,dp=0.2cm,wd=\textwidth,leftskip=.3cm]{author in head/foot}%
        \begin{minipage}[c]{5cm}%
        \usebeamerfont{author in head/foot}
        \end{minipage}\hfill% 
        \begin{minipage}{6cm}
        \hfill\includegraphics[height=.5cm]{AMCGFlow-long}
        \end{minipage}
\end{beamercolorbox}%
}
\begin{frame}
\titlepage
\end{frame}

\begin{frame}{Introduction}

This presentation will cover:
\vspace{1ex}
\begin{description}
 \item[$\cdot$] Some reasons your simulation can fail
 \item[$\cdot$] Sources of useful data
 \item[$\cdot$] What can cause numerical instability?
 \item[$\cdot$] Simplifying your problem
 \item[$\cdot$] (Occasionally) helpful third party software 
 \item[$\cdot$] HELP!
\end{description}
\end{frame}

%%%%%%%%%%%%%%%%%%%%%%%%%%%%%%%%%%%%%%%%%%%%%%%%%%%%%%%%%

\begin{frame}{Some reasons that your simulation can fail}

Roughly speaking, the reason that your simulation can failed will be placed in one of the following categories:
\vspace{1ex}

\begin{description}
 \item[$\cdot$] Bad environment variables, missing/out-of-date dependent software or modules
 \item[$\cdot$] Option errors, faulty Python scripts
 \item[$\cdot$] Under/over/badly constrained initial and/or boundary conditions
 \item[$\cdot$] Numerical instability
 \item[$\cdot$] Partitioning, adaptivity, field projection
\end{description}

\end{frame}

%%%%%%%%%%%%%%%%%%%%%%%%%%%%%%%%%%%%%%%%%%%%%%%%%%%%%%%%%

\begin{frame}{Sources of useful information}

Information that can help you diagnose the failure can be gathered from:
\vspace{1ex}

\begin{description}
 \item[$\cdot$] log file (``{\tt -l -v3}'')
 \item[$\cdot$] error file (including backtrace)
 \item[$\cdot$] stat file
 \item[$\cdot$] pbs output (if running on a HPC)
 \item[$\cdot$] ``matrixdump'' and ``core'' files
 \item[$\cdot$] vtus
\end{description}

It is also possible to re-run your simulation with extra diagnostic fields, increase your dump-rate, 
and re-run the simulation with a Fluidity binary compiled with debugging enabled.

\end{frame}

%%%%%%%%%%%%%%%%%%%%%%%%%%%%%%%%%%%%%%%%%%%%%%%%%%%%%%%%%

\begin{frame}{``My simulation broke in 10 seconds flat!''}

If your simulation broke very very quickly 
then it was more than likely due to one of the following reasons:
\vspace{1ex}

\begin{description}
 \item[$\cdot$] Bad environment variables, missing/out-of-date dependent software or modules
 \item[$\cdot$] Option errors, faulty Python scripts
\end{description}

A common mistake, for instance, is having a mis-directed PYTHONPATH environment variable.  
You can check this by typing ``{\tt echo \$PYTHONPATH}'' into your terminal window 
(the list of paths should include {\tt [Fluidity directory]/python})

\vspace{1ex}
If you're suspicious that your Fluidity build isn't working correctly, navigate to the Fluidity directory and type ``{\tt make unittest}''.

\vspace{1ex}
For an option error, then hopefully the error message should give useful advice on what to correct,
but if all else fails, backing-up your {\tt .flml} file and progressively removing fields can help. 
Also, using the {\tt diff} and {\tt meld} commands to compare against working {\tt .flml} files.
\end{frame}

%%%%%%%%%%%%%%%%%%%%%%%%%%%%%%%%%%%%%%%%%%%%%%%%%%%%%%%%%

\begin{frame}{What can cause numerical instability?}

Thankfully, for most causes of failure, the error messages given by Fluidity are fairly self-explanatory.
But, unfortunately, the cause of numerical instability is often a little more tricky to pin-point.
The symptom described within an error message may not directly relate the underlying cause of failure.

\vspace{1ex}
For this reason, the best advice is to sanity check the various factors that can cause numerical instability:
\begin{description}
 \item[$\cdot$] Timestep, Courant Number, and temporal discretisation
 \item[$\cdot$] Spatial discretisation and element pair
 \item[$\cdot$] Mesh quality, Condition Number
 \item[$\cdot$] Mesh resolution, interpolation error, Grid Reynolds Number
 \item[$\cdot$] Field stabilisation, upwinding/slope-limiting
 \item[$\cdot$] Strong/weak boundary conditions
 \item[$\cdot$] Preconditioner, iterative solver, convergence criteria
\end{description}

\end{frame}

%%%%%%%%%%%%%%%%%%%%%%%%%%%%%%%%%%%%%%%%%%%%%%%%%%%%%%%%%

\begin{frame}{Simplifying your simulation}

Once you have extracted any useful data from a broken simulation
it is important that you make any further progress from a simpler simulation that {\it did} solve successfully.
If no such simulation exists, then you need to create one!

\vspace{1ex}
There are several ways in which you can simplify your simulation:

\begin{description}
 \item[$\cdot$] Reduce the number of spatial dimensions
 \item[$\cdot$] Simplify the geometry 
 \item[$\cdot$] Smooth BCs with discontinuities in space or time
 \item[$\cdot$] If you have a viscosity field, then set that to a value that gives you a Reynolds Number $\simeq 1000$
 \item[$\cdot$] Remove any turbulence models and prescribe viscosity (as above)
 \item[$\cdot$] Progressively remove prognostic fields
 \item[$\cdot$] Remove adaptivity, lower the number of partitions
\end{description}

Then, add complexity back in \emph{\color{red}ONE STEP AT A TIME}

\end{frame}

%%%%%%%%%%%%%%%%%%%%%%%%%%%%%%%%%%%%%%%%%%%%%%%%%%%%%%%%%

\begin{frame}{Monitoring the progress of a simulation}

If you are suspicious that your simulation may crash and you want to monitor its progress,
then the following commands can be useful:

\vspace{1ex}
\begin{description}
 \item[] {\tt statplot *.stat} (press ``{\tt r}'' to refresh)
 \item[] {\tt tail -f fluidity.log-0}
 \item[] {\tt grep ``n/o iterations'' fluidity.log-0}
 \item[] {\tt grep reason fluidity.log-0}
\end{description}

Note that for the last three to work you need to be logging with a verbosity of two or more, 
i.e. {\tt ./fluidity -v2 -l my.flml}

\end{frame}

%%%%%%%%%%%%%%%%%%%%%%%%%%%%%%%%%%%%%%%%%%%%%%%%%%%%%%%%%

\begin{frame}{(Occasionally) helpful third-party software}

If you believe there might be a bug in the code
then your first step should be to contact the development team and submit a bug report.

\vspace{1ex}
The following third-party software can then be useful for slowly stepping through the simulation solve to determine the root cause:
\vspace{1ex}

\begin{description}
 \item[$\cdot$] Valgrind (checking for memory issues)
 \item[$\cdot$] GDB (a free Open Source de-bugger)
 \item[$\cdot$] DDT (a commercial alternative) 
\end{description}

Similarly ``vtune'' can be useful in determining which lines of code are taking up the most time in the solve.

\end{frame}

%%%%%%%%%%%%%%%%%%%%%%%%%%%%%%%%%%%%%%%%%%%%%%%%%%%%%%%%%

\begin{frame}{HELP!!!}

If you've tried and failed to fix your problem, then there is a troubleshooting section to the Manual that's worth consulting.

\vspace{1ex}
But finally, Fluidity has a very eager and responsive development community.  
We can be contacted either through the mailing list ({\tt mailman.ic.ac.uk/mailman/listinfo/fluidity})
or through IRC ({\tt \# amcg}).

\vspace{1ex}
It would be helpful for you to include information about:
\begin{description}
 \item[$\cdot$] What simulation are you trying to run?
 \item[$\cdot$] What momentum equation are you solving?
 \item[$\cdot$] What discretisation are you using?
 \item[$\cdot$] What similar simulations do you have that {\it did} work?
 
\end{description}
\end{frame}
%%%%%%%%%%%%%%%%%%%%%%%%%%%%%%%%%%%%%%%%%%%%%%%%%%%%%%%%%

\begin{frame}
\begin{center}
{\Huge Questions?}
\end{center}
\end{frame}

\end{document}
