%-- Add sections and your outline will be created automatically --%
\section{Tsunami}

% Frame starts a new slide
\begin{frame}
    \frametitle{Hokkaido-Nansei-Oki tsunami}
\begin{minipage}[]{0.5\linewidth} 
\begin{itemize}
\item Okushiri island, Japan, 1993. 
\item Runup height of up to 30m.
\item Simulation based on a 1:400 laboratory setup.
\item Uses the free-surface and wetting and drying functionality of Fluidity.
\end{itemize}
\end{minipage}
\hspace{0.5cm}
\begin{minipage}[]{0.4\linewidth} 
\begin{figure}
\begin{center}
\includegraphics[width=\textwidth]{hokkaido-nansei-oki_tsunami/MonaiValleyDomainWithInputWave2_png.pdf}
\end{center}
\caption{The domain and the three gauge stations.}\label{fig:monai_inputwave}
\end{figure}
\end{minipage}
\end{frame}

\begin{frame}
    \frametitle{Hokkaido-Nansei-Oki tsunami}
\begin{figure}
\begin{center}
\includegraphics[width=0.7\textwidth]{hokkaido-nansei-oki_tsunami/MonaiValley_C_p1p1_nu0_01_kmkstab_drag0_002_butcircularoundisland0_2-crop-crop_final2.pdf}
\caption{The numerical and experimental results at the three gauge stations.}\label{fig:monai_results}
\end{center}
\end{figure}
% end my slide
\end{frame}


\begin{frame}
    \frametitle{Hokkaido-Nansei-Oki tsunami - Exercises}
  \begin{itemize}
    \item Add more detectors.\newline
    \item Check how increasing the wetting and drying threshold parameter affects the results.\newline
    \item Try changing the viscosity value (How does it affect the inundation of the tsunami event?).
  \end{itemize}
% end my slide
\end{frame}


